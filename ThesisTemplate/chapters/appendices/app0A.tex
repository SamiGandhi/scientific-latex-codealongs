\section{Additional plots}
\label{app:app01}

\begin{figure}[ht]
    \centering
	\begin{subfigure}[b]{0.24\textwidth}
        \includegraphics[width=\textwidth]{chapters/chapter05/fig05/Model_Seasons_Evaluation_Common_Sandpiper_Yolo.png}
        \caption{Common Sandpiper}
        \label{fig:yolo2_appendix}
    \end{subfigure}
    \hfill
    \begin{subfigure}[b]{0.24\textwidth}
        \includegraphics[width=\textwidth]{chapters/chapter05/fig05/Model_Seasons_Evaluation_Common_Shelduck_Yolo.png}
        \caption{Common Shelduck}
        \label{fig:yolo3_appendix}
    \end{subfigure}
    \hfill
    \begin{subfigure}[b]{0.24\textwidth}
        \includegraphics[width=\textwidth]{chapters/chapter05/fig05/Model_Seasons_Evaluation_Plumed_Whilstling_Duck_Yolo.png}
        \caption{Plumed Whistling Duck}
        \label{fig:yolo5_appendix}
    \end{subfigure}
    \hfill
    \begin{subfigure}[b]{0.24\textwidth}
        \includegraphics[width=\textwidth]{chapters/chapter05/fig05/Model_Seasons_Evaluation_Wood_Stork_YoLo.png}
        \caption{Wood Stork}
        \label{fig:yolo6_appendix}
    \end{subfigure}

    \caption*{Figure A.1.1: The values of counted objects YOLOv8 models for the differents video sequences across the seasons.}
    \label{fig:yolo_count_appendix}
\end{figure}
%\subsubsection{Counted Objects Using the Combination of The Two Models}

\begin{figure}[ht]
    \centering
	\begin{subfigure}[b]{0.24\textwidth}
        \includegraphics[width=\textwidth]{chapters/chapter05/fig05/Model_Seasons_Evaluation_Common_Sandpiper_Ltce.png}
        \caption{Common Sandpiper}
        \label{fig:lcte2_appendix}
    \end{subfigure}
    \hfill
    \begin{subfigure}[b]{0.24\textwidth}
        \includegraphics[width=\textwidth]{chapters/chapter05/fig05/Model_Seasons_Evaluation_Common_Shelduck_Ltce.png}
        \caption{Common Shelduck}
        \label{fig:lcte3_appendix}
    \end{subfigure}
    \hfill
    \begin{subfigure}[b]{0.24\textwidth}
        \includegraphics[width=\textwidth]{chapters/chapter05/fig05/Model_Seasons_Evaluation_Plumed_Whilstling_Duck_Ltce.png}
        \caption{Plumed Whistling Duck}
        \label{fig:lcte5_appendix}
    \end{subfigure}
    \hfill
    \begin{subfigure}[b]{0.24\textwidth}
        \includegraphics[width=\textwidth]{chapters/chapter05/fig05/Model_Seasons_Evaluation_Wood_Stork_Ltce.png}
        \caption{Wood Stork}
        \label{fig:lcte6_appendix}
    \end{subfigure}

    \caption*{Figure A.1.2: The values of counted objects using the combination of the two models for the different video sequences across the seasons.}
    \label{fig:lcet_count_appendix}
\end{figure}


%\subsubsection{False Positives}


\begin{figure}[ht]
    \centering
	\begin{subfigure}[b]{0.24\textwidth}
        \includegraphics[width=\textwidth]{chapters/chapter05/fig05/Model_Seasons_Evaluation_Common_Sandpiper_FP.png}
        \caption{Common Sandpiper}
        \label{fig:false_postives2_appendix}
    \end{subfigure}
    \hfill
    \begin{subfigure}[b]{0.24\textwidth}
        \includegraphics[width=\textwidth]{chapters/chapter05/fig05/Model_Seasons_Evaluation_Common_Shelduck_FP.png}
        \caption{Common Shelduck}
        \label{fig:false_postives3_appendix}
    \end{subfigure}
    \hfill
    \begin{subfigure}[b]{0.24\textwidth}
        \includegraphics[width=\textwidth]{chapters/chapter05/fig05/Model_Seasons_Evaluation_Plumed_Whilstling_Duck_FP.png}
        \caption{Plumed Whistling Duck}
        \label{fig:false_postives5_appendix}
    \end{subfigure}
    \hfill
    \begin{subfigure}[b]{0.24\textwidth}
        \includegraphics[width=\textwidth]{chapters/chapter05/fig05/Model_Seasons_Evaluation_Wood_Stork_FP.png}
        \caption{Wood Stork}
        \label{fig:false_postives6_appendix}
    \end{subfigure}

    \caption*{Figure A.1.3: The values of False Positive for the different video sequences across the seasons.}
    \label{fig:false_postives_appendix}
\end{figure}


%\subsubsection{Average Confidence}


\begin{figure}[ht]
    \centering
	\begin{subfigure}[b]{0.24\textwidth}
        \includegraphics[width=\textwidth]{chapters/chapter05/fig05/Model_Seasons_Evaluation_Common_Sandpiper_Confidence.png}
        \caption{Common Sandpiper}
        \label{fig:confidence_seasons2_appendix}
    \end{subfigure}
    \hfill
    \begin{subfigure}[b]{0.24\textwidth}
        \includegraphics[width=\textwidth]{chapters/chapter05/fig05/Model_Seasons_Evaluation_Common_Shelduck_Confidence.png}
        \caption{Common Shelduck}
        \label{fig:confidence_seasons3_appendix}
    \end{subfigure}
    \hfill
    \begin{subfigure}[b]{0.24\textwidth}
        \includegraphics[width=\textwidth]{chapters/chapter05/fig05/Model_Seasons_Evaluation_Plumed_Whilstling_Duck_Confidence.png}
        \caption{Plumed Whistling Duck}
        \label{fig:confidence_seasons5_appendix}
    \end{subfigure}
    \hfill
    \begin{subfigure}[b]{0.24\textwidth}
        \includegraphics[width=\textwidth]{chapters/chapter05/fig05/Model_Seasons_Evaluation_Wood_Stork_Confidence.png}
        \caption{Wood Stork}
        \label{fig:confidence_seasons6_appendix}
    \end{subfigure}

    \caption*{Figure A.1.4: Average confidence in detecting objects for the video sequences across different seasons.}
    \label{fig:confidence_seasons_appendix}
\end{figure}





\section{VidSim Simulation Tool – Technical Documentation}
\label{appendix:vidsim}

\subsection{Introduction}
VidSim is a simulation platform developed to support the evaluation of video encoding and transmission strategies within Multimedia Wireless Sensor Networks (MWSNs). It complements the system presented in this thesis by providing a controlled environment for measuring performance under a wide range of conditions. The tool is implemented in \texttt{Python} and offers a modular, user-friendly environment for encoding simulation, energy modeling, and quality evaluation.

\subsection{Software Architecture}
VidSim follows a modular design, enabling independent development and testing of core components. The system is composed of the following modules:

\begin{itemize}
  \item \textbf{Video Capture Module}: Acquires frames from video files or camera streams.
  \item \textbf{ROI Processing Module}: Applies block-based Region of Interest (ROI) detection and adaptive quantization.
  \item \textbf{Simulation Control Module}: Handles dynamic simulation parameters like frame rate, GOP size, and quality factors.
  \item \textbf{Performance Analysis Module}: Computes quality metrics and energy consumption statistics.
\end{itemize}

\begin{figure}[H]
  \centering
  \includegraphics[width=0.8\textwidth]{chapters/appendices/Flow_Chart_RoI_Frame.png}
  \caption*{Figure A.2.1: VidSim system architecture (recreated from paper).}
  \label{fig:vidsim_architecture}
\end{figure}

\subsection{Performance Metrics}

VidSim supports a comprehensive evaluation of video quality using standard and perceptual metrics:

\subsection*{PSNR (Peak Signal-to-Noise Ratio)}
\begin{equation}
  \text{PSNR} = 10 \cdot \log_{10} \left( \frac{MAX_I^2}{MSE} \right)
\end{equation}

\subsubsection*{SSIM (Structural Similarity Index)}
\begin{equation}
  SSIM(x, y) = \frac{(2\mu_x\mu_y + C_1)(2\sigma_{xy} + C_2)}{(\mu_x^2 + \mu_y^2 + C_1)(\sigma_x^2 + \sigma_y^2 + C_2)}
\end{equation}

\subsubsection*{BRISQUE}
BRISQUE is a no-reference image quality assessment metric based on natural scene statistics.

\subsection{Energy Consumption Model}

VidSim tracks energy use across all processing stages using the following equations:

\subsubsection*{Capture Energy}
\begin{equation}
  E_{\text{cap}} = P_{\text{cap}} \cdot t_{\text{cap}}
\end{equation}

\subsubsection*{Encoding Energy}
\begin{equation}
  E_{\text{enc}} = P_{\text{enc}} \cdot t_{\text{enc}}
\end{equation}

\subsubsection*{Transmission Energy}
\begin{equation}
  E_{\text{trans}} = P_{\text{trans}} \cdot t_{\text{trans}}
\end{equation}

\subsubsection*{Total Energy}
\begin{equation}
  E_{\text{total}} = E_{\text{cap}} + E_{\text{enc}} + E_{\text{trans}}
\end{equation}

\subsection{Simulation Workflow}

\begin{figure}[H]
  \centering
  \includegraphics[width=0.8\textwidth]{chapters/chapter04/fig04/Flow_Chart_Standard_Frame.png}
  \caption*{Figure A.2.2: VidSim simulation workflow (adapted from the paper).}
  \label{fig:vidsim_workflow}
\end{figure}

The typical simulation process includes:
\begin{enumerate}
  \item Parameter configuration
  \item Frame acquisition
  \item ROI-based encoding
  \item Transmission modeling
  \item Performance evaluation
\end{enumerate}

\subsection{Parameter Management}

Simulation parameters in VidSim are configurable via both scripts and graphical interfaces. Dynamic adjustment during runtime is supported.

\begin{table}[H]
  \centering
  \caption*{Table A.2.1: Key configurable parameters in VidSim.}
  \label{tab:vidsim_parameters}
  \begin{tabular}{|l|p{10cm}|}
    \hline
    \textbf{Parameter} & \textbf{Description} \\
    \hline
    Frame Rate & Number of frames per second \\
    Resolution & Frame dimensions (e.g., 640×480) \\
    GOP Size & Number of frames between successive I-frames \\
    Quality Factor & Compression level for ROI and non-ROI blocks \\
    Encoding Method & Type of transform (DCT, entropy coding, etc.) \\
    Bitrate / BPP & Controls video data size and quality \\
    Environmental Factors & Humidity (H), Vegetation Density (V), Distance (D) \\
    \hline
  \end{tabular}
\end{table}

\subsection{Petri Net Representation}

The tool's internal logic is modeled using a Petri net structure to represent token transitions across modules and ensure modular concurrency.

\begin{figure}[H]
  \centering
  \includegraphics[width=0.8\textwidth]{chapters/appendices/petri_net_rep.png}
  \caption*{Figure A.2.3: Petri Net model of VidSim simulation flow.}
  \label{fig:vidsim_petri_net}
\end{figure}

\subsection*{Conclusion}
This appendix presented the internal structure, configuration, and processing model of the VidSim tool developed for simulation-based validation of the proposed wildlife monitoring framework. The tool enables precise evaluation under diverse environmental conditions and compression strategies, complementing the real-world tests outlined in the thesis.
